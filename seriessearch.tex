\documentclass{article}
\usepackage{graphicx} % Required for inserting images
\usepackage{amsmath,amssymb}
\usepackage{enumerate}
\title{Series Search}

\begin{document}

\maketitle

\section{Introduction}
This project is intended to study computational applications in analysis. 
Inspired by the elusive value of $\zeta (3)$ (Apery's constant), 
I wanted to see a potential computational approach to developing closed forms for infinite series. 
While Apery's constant is unknown, I intend to derive the famous result $\zeta (2) = \frac{{\pi}^2}{6}$ 
through the algorithm below. 

\subsection{Formal Development}

Let $S$ be a sequence whose sum converges absolutely to a value $\delta$, with terms $\Delta_n$. 
Further, let $S'$ be a sequence that converges absolutely to a value $\delta'$, with terms $\Delta_n'$. 

\subsection{Basic Definitions} 

\begin{enumerate}[i]
    \item $Ord_R(S)$ is the sequence $K$, such that $S$ is reordered by a relation $R$.
    \item $Abs(S)$ is the sequence $K$, such that $K = (|\Delta_i|,|\Delta_{i+1}|,|\Delta_{i+2}|,|\Delta_{i+3}|,... )$
    \item $S'$ is an undercompensate of $S$ if there exists a term $\Delta_k \neq \Delta_i$, for all appropriate terms in $S'$, 
          $\Delta_i$. We denote this with $S \sqsubset S'$
    \item $S'$ is an overcompensate of $S$ if there exists a term $\Delta_i' \neq \Delta_k$, for all appropriate terms in $S$, 
          $\Delta_k$. We denote this with $S \sqsupset S'$
    \item $S'$ is said to be \textbf{total} to $S$ if $Ord(Abs(S')) = Ord(Abs(S))$ We denote this with $S \equiv S'$. 
          Considering the sums of the respective series are absolutely convergent, one can see that the OrdAbs 
          operator on $S'$ and $S$ is purely a formality, and not necessary. $S'$ is said to be \textbf{equal} to $S$ 
          if and only if $S' \equiv S$ and $\delta' = \delta$
  \end{enumerate}
\textbf{Note:  one may rewrite the above definitions in the language of functions, that is through injections, 
surjections, and bijections on the set of terms $\{x | x = \Delta_n\}$ and $\{x | x = \Delta_n'\}$}
\newpage
\textbf{-Abuse of Notation, Additional Definitions-}
\begin{enumerate}[i]
    \item Let $A$ and $B$ be sequences whose sum absolutely converges, with terms $a_n$ and $b_n$. 
          The sequence union of $A$ and $B$, denoted $A \cup B$ is the sequence $(a_1, b_1, a_2, b_2, ...)$. 
    \item Let $A$ and $B$ be absolutely convergent sequences with terms $a_n$ and $b_n$.  
          The sequence intersection of $A$ and $B$, denoted $A \cap B$, is the sequence $(c_i, c_{i+1}, c_{i+2}...)$, 
          where $c_i = a_i$ where $a_i = b_i$.
    \item Let $A$ and $B$ be absolutely convergent sequences with terms $a_n$ and $b_n$.  
          The sequence minus of $A$ and $B$, denoted $A \setminus B$, is the sequence $(c_i, c_{i+1}, c_{i+2}...)$, 
          where $c_i = a_i, b_i,$ where $a_i \neq b_i$.
    \item Let $A_1, A_2, A_3... A_n$ be sequences whose sum converges absolutely. Then, $\mathcal{A} = \{A_i\}$ 
          is said to be partitions of $S$ if $A_i$ are undercompensates of $S$ and 
          $Ord_{\leq}(Abs(\bigcup\limits_{i=1}^{n} A_{i})) = Ord_{\leq}(Abs(S))$.
\end{enumerate}



\subsection{Series Encoding}

We formalize the notion of encoding a series. The \textbf{encoding} of a series is a tuple 
$\mathcal{E} = (\rho_{0}, \mathcal{T}, N)$, where $\rho_{0}$ denotes the initial value of a series, 
$\mathcal{T}$ denotes a transform $\mathcal{T}: \mathbb{Q} \rightarrow \mathbb{Q}$ (for this purpose, 
a series is over the rationals), and $N$ which denotes the number of transition states in the transform for 
the summation. The transform $\mathcal{T}$ should be defined such that $\mathcal{T}(\rho_{i}) = \rho_{i+1}$.
The n'th iteration of $\mathcal{T}$ on $\rho_{i}$, (i.e. $\mathcal{T}(\mathcal{T}(\mathcal{T}...(\rho_{i})...))$) will 
be denoted as $\mathcal{T}^n(\rho_{i})$.

\subsection{Series Decomposition}

We formalize the notion of decomposing a series. Given a series, $S$ $(=\Delta_n)$, $S$ may be decomposed 
into a partition $\mathcal{A}$ and encoded as $\mathcal{Z}$.

\subsubsection{Example} 

Consider a geometric series, $G$, given by \[ G = \sum_{n=1}^{\infty} (\frac{1}{3})^{n-1} \]
A valid partition for G is given by:\\
$\mathcal{A} = \{A_1, A_2\}$, where\\ \[A_1 = \sum_{n=0}^{\infty} (\frac{1}{3})^{2n} , A_2 = \sum_{n=0}^{\infty} (\frac{1}{3})^{2n+1} \]
Encoded as: $\text{Enc}(\mathcal{A}) = \mathcal{Z} = 
\{(1, \mathcal{T}_0(s) = \frac{1}{9}s, |\mathbb{N}|),(\frac{1}{3}, \mathcal{T}_1(s) = \frac{1}{9}s, |\mathbb{N}|) \}$

\subsection{Series Searching}

Theorem 1.1
Let $S$ be an absolutely convergent series with term $\Delta_n$.
Let $\mathcal{U}$ be the set of all absolutely converging series with term $u_{i_n} \in \mathcal{U}$. 
Let $\mathcal{P}$ be the set of all encodings of series in the partitions of $\mathcal{U}$. 
Construct a directed complete graph where $G = (U, V)$, where the set of vertices, $V = \{p \in P\}$. 
A set $P \subseteq \mathcal{P}$ is said to $complete$ $S$ if there exists 
\end{document}
